\chapter{引言}
%\rhead{引言}
\section{研究背景和意义}

由于生物实验的局限性,传统的微生物基因组学经常关注于单个人的细菌基因组。然而,环境中的微生物基因组通常会互相产生影响。
例如,人类中的微生物已被证明与常见疾病有关
如炎性肠病(IBD)的[1]和胃肠紊乱[2]。
宏基因组学(环境基因组学或生态基因组学)是直接从环境样本,例如人体肠道,土壤,空气中的尘埃中直接研究遗传物的学科。快速发展
随着新一代测序(NGS)技术的快速发展,我们可以直接对混合环境的DNA样品获得的多个物种进行测序。
宏基因组序列来自多个微生物基因组,并且通常
大多数宏基因组序列所在的物种名是未知的。
宏基因组分析的关键步骤就是将同一物种的DNA片段聚在一起。
这个任务被称为宏基因组序列的归类[3]。迄今为止,研究人员已经提出了大量的计算方法来解决这个问题。这些方法可以粗略的归为两类:
基于相似度和基于组成成分。

基于相似度的方法首先将元基因组序列比对到已知的基因组,然后根据比对结构进行归类。其中一个典型的方法是MEGAN[4].

显然,如果没有已知的微生物基因组,这个方法不能用。基于组成成分的方法通常采用
监督技术 将序列分配到不同的组。特征是直接从核苷酸序列从抽取的,包括寡核苷酸的频率
GC含量,密码子的使用等等。到现在为止,SVM [5],朴素贝叶斯[6],KNN [7],Interpolated 马尔可夫模型[8]等已经被用来对宏基因组序列进行归类。然而这些方法的性能仍然在很大程度上依赖于作为训练样本的基因组。
为了克服这些方法的缺点以上,研究人员又提出非监督或半监管技术来处理未知物种的宏基因组数据。 Wu等人[9]提出了一种称为AbundanceBin方法从序列中抽取K-mer,并利用K-mer覆盖率进行归类,以达到区分物种丰度比差异明显的数据的效果。然而,当数据集的物种丰度比相同时,AbundanceBin表现差强人意。Leung等人 [10]开发了MetaClus-3.0方法
通过采用4-mer来构建特征向量。它采用K-meadian算法将序列分成小的簇,然后将较小的簇合并成大簇以使物种丰度较低的序列归类到孤立的簇。MetaCluster3.0 在序列长为1000bp的均匀和不均匀物种上的表现都优于AbundanceBin。随后,Wang等人提出了MetaCluster3.0的两个改进算法,MetaCluster4.0 [11] 和MetaCluster5.0 [12],来处理
处理短序列。MetaCluster4.0基于序列重叠性,将长度小于500bp的短序列组装为长序列
但是,它处理低丰度的序列。MetaCluster5.0作为MetaCluster4.0的改进版,能够处理低丰度数据。2014年APBC会议上,WangYi等研究人员开发
MetaCluster-TA [13],一个通过装配和归类来进行注释的序列分类注释工具。它先将短序列装配成长的“虚拟重叠群”,并
然后应用类似于MetaCluster5.0的方法来对这些重叠群和序列进行聚类,
并最后将所有的簇进行归类。这一系列的MetaCluster
算法可以自动确定簇的数目,这
对于真实数据中绝大多数序列的物种名未知时是至关重要的。



\section{本文结构}
本文的组织结构如下:在第二章中我们介绍目前在图可达性查询领域的一些相关工作。在第三章中,将介绍本文算法的背景知识、问题定义和解决方法的概要介绍。在第四章中,我们提出了\pphop 索引建立算法,索引优化算法和查询流程优化,并通过实验对我们的索引方法及其查询效率进行验证。在第五章中,我们针对稀疏图提出了一种更加优化的\ppmuihop 索引方法,并通过实验,对该算法在稀疏图上的数据特性进行分析。在第六章中,我们对本文进行总结,并对将来可以进一步改进的方向进行展望。
