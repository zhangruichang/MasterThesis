\chapter*{Abstract}
\addcontentsline{toc}{chapter}{Abstract}
\rhead{Abstract}
%please write the english abstract here

Due to the massive volume of graph data from a wide range of recent
applications, such as social network, bioinformatics and XML data. At the same time, much more efficient and powerful resources required to process numerous queries.
It is becoming economically appealing to outsource graph data to a third-party
service provider (\SP) to provide query services. However, \SP\ cannot always
be trusted. Hence, data owners and query clients may prefer not to expose their
data graphs and queries.

Reachiablity query is one of the most fundamental query in database area. This paper studies privacy-preserving query services
for reachability query where both clients' queries and the structural information of the owner's data are protected.
According to different types of graph data, we propose two different index method base on \hop\   labeling, named privacy-preserving \hop\  labeling (\pphop) and privacy-preserving minimum unified intesection \hop\  (\ppmuihop) . In these two methods, the reachability queries are
computed in an encrypted domain and the input and output sizes of any queries
are indistinguishable. We analyze the security of these two method with respect to ciphertext
only and size based attacks. We verify the performance of \pphop\ and  \ppmuihop\ with an
experimental study on both synthetic and real-world datasets.
\bigskip
\bigskip

\noindent{\bf Keywords}: Graph data, Reachability Query, \hop , Index, Privacy Preserving Method, Query Service%key word

 \noindent{\bf Classification Code}: TP311
